\documentclass[11pt, a4paper, leqno]{article}
\usepackage{a4wide}
\usepackage[T1]{fontenc}
\usepackage[utf8]{inputenc}
\usepackage{float, afterpage, rotating, graphicx}
\usepackage{epstopdf}
\usepackage{longtable, booktabs, tabularx}
\usepackage{fancyvrb, moreverb, relsize}
\usepackage{eurosym, calc}
% \usepackage{chngcntr}
\usepackage{amsmath, amssymb, amsfonts, amsthm, bm}
\usepackage{caption}
\usepackage{mdwlist}
\usepackage{xfrac}
\usepackage{setspace}
\usepackage[dvipsnames]{xcolor}
\usepackage{subcaption}
\usepackage{minibox}
% \usepackage{pdf14} % Enable for Manuscriptcentral -- can't handle pdf 1.5
% \usepackage{endfloat} % Enable to move tables / figures to the end. Useful for some
% submissions.

\usepackage[
    natbib=true,
    bibencoding=inputenc,
    bibstyle=authoryear-ibid,
    citestyle=authoryear-comp,
    maxcitenames=3,
    maxbibnames=10,
    useprefix=false,
    sortcites=true,
    backend=biber
]{biblatex}
\AtBeginDocument{\toggletrue{blx@useprefix}}
\AtBeginBibliography{\togglefalse{blx@useprefix}}
\setlength{\bibitemsep}{1.5ex}
\addbibresource{../../paper/refs.bib}

\usepackage[unicode=true]{hyperref}
\hypersetup{
    colorlinks=true,
    linkcolor=black,
    anchorcolor=black,
    citecolor=NavyBlue,
    filecolor=black,
    menucolor=black,
    runcolor=black,
    urlcolor=NavyBlue
}


\widowpenalty=10000
\clubpenalty=10000

\setlength{\parskip}{1ex}
\setlength{\parindent}{0ex}
\setstretch{1.5}


\begin{document}

\title{Agricultural diversity, Structural Change and long-run development: Evidence from US\thanks{Renuka Venkataramani, Bonn University. Email: \href{mailto:s6revenkQuni-bonn.de}{\nolinkurl{s6revenkQuni-bonn [dot] de}}.}}

\author{Renuka Venkataramani}


\maketitle


\begin{abstract}
    This project replicates a paper title 'Agricultural diversity, Structural Change and long-run development: Evidence from US.
    This project tries to replicate the analysis result using Python
    \citet{fiszbein2022agricultural}
\end{abstract}

\clearpage


\section{Introduction} % (fold)
\label{sec:introduction}

One of the unique feature of this file is the use of shape files. Shape files run with the extensions shp, shx, dbf, etc and are read using GIS software.
One such package is geopandas and this project using geopandas and matplotlib.
\citet{GaudeckerEconProjectTemplates}.




\begin{figure}[H]

    \centering
    \includegraphics[width=0.85\textwidth]{https://www.dropbox.com/s/n1pb9ofj0ednne6/mapfile_png.png?dl=1}

    \caption{\emph{Python:} This is the map produced by merging the data column share of population in manufacturing
    inductry with the shapefile}
    \label{fig:Mapfile: Share of population in the manufacturing industry}

\end{figure}

I also created an interactive map using folium package to show the same. 
\url{https://www.dropbox.com/s/xgxo4l28m17g5hp/plot.html?dl=0}


\begin{tabular}{rlrrrr}
    \toprule
    \hline
        & Product            &   Mean &   Maximum &   Dominant &   >50% \\
    \hline
      0 & wheat              &  10.16 &     92.85 &       8.4  &   0.44 \\
      1 & rye                &   0.92 &     15.54 &       0    &   0    \\
      2 & corn               &  23.13 &     98.89 &      43.66 &  10.43 \\
      3 & oats               &   3.86 &     23.83 &       0    &   0    \\
      4 & rice               &   0.18 &     46.35 &       0    &   0    \\
      5 & tobacco            &   1.85 &     36.41 &       0    &   0    \\
      6 & cotton             &  15.83 &     94.11 &      20.04 &  13.51 \\
      7 & wool               &   0.99 &    100    &       0    &   0.22 \\
      8 & peas               &   0.71 &     22.3  &       0    &   0    \\
      9 & potatoes           &   3.07 &     59.87 &       0    &   0.11 \\
     10 & sweet_potatoes     &   1.22 &     29.16 &       0    &   0    \\
     11 & barley             &   0.44 &     10.52 &       0    &   0    \\
     12 & buckwheat          &   0.56 &     15.31 &       0    &   0    \\
     13 & wine               &   0.02 &      4.68 &       0    &   0    \\
     14 & butter             &   4.18 &     25.36 &       0    &   0    \\
     15 & cheese             &   0.7  &     35.69 &       0    &   0    \\
     16 & hay                &  11.77 &     73.97 &      17.24 &   1.65 \\
     17 & clover             &   0.3  &     12.93 &       0    &   0    \\
     18 & grass_seed         &   0.15 &      9.39 &       0    &   0    \\
     19 & hops               &   0.17 &     19.51 &       0    &   0    \\
     20 & dew_rotted_hemp    &   0.26 &     64.76 &       0    &   0.05 \\
     21 & water_rotted_hemp  &   0.02 &      7.14 &       0    &   0    \\
     22 & other_hemp         &   0.09 &     21.68 &       0    &   0    \\
     23 & flax               &   0.04 &      4.77 &       0    &   0    \\
     24 & flax_seed          &   0.04 &      3.34 &       0    &   0    \\
     25 & silk_coccoon       &   0.01 &      2.86 &       0    &   0    \\
     26 & maple_sugar        &   0.22 &     36.26 &       0    &   0    \\
     27 & cane_sugar         &   0.69 &     46.83 &       0    &   0    \\
     28 & maple_molasses     &   0.06 &      7.41 &       0    &   0    \\
     29 & cane_molasses      &   0.16 &     10.5  &       0    &   0    \\
     30 & sorghum_molasses   &   0.25 &      7    &       0    &   0    \\
     31 & beeswax            &   0.02 &      1.03 &       0    &   0    \\
     32 & honey              &   0.23 &      8.38 &       0    &   0    \\
     33 & orchads            &   1.16 &     24.01 &       0    &   0    \\
     34 & market_garden      &   0.89 &     98.14 &       0    &   0.22 \\
     35 & animal_slaughtered &  12.76 &     95.16 &       4.34 &   0.55 \\
    \hline
    \endrule
    \caption {Table 1 produced from the analysis gives the Mean, Max, Dominance and <50 \% of each crops}
    \end{tabular}





% section introduction (end)



\setstretch{1}
\printbibliography
\setstretch{1.5}


% \appendix

% The chngctr package is needed for the following lines.
% \counterwithin{table}{section}
% \counterwithin{figure}{section}

\end{document}
